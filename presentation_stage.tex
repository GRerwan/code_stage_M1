\documentclass[8pt]{beamer}


% Utilisation du thème Darmstadt
\usetheme{Darmstadt}

% Packages nécessaires pour la couleur
\usepackage{xcolor}
\usepackage{siunitx}
\usepackage{amsmath,amssymb,amsfonts}
\usepackage{listings}
\usepackage{graphicx}
\usepackage{textcomp}
\usepackage{url}
\usepackage{comment}
\def\UrlFont{\rmfamily}
\usepackage{booktabs}
\usepackage{multicol}
\usepackage{float} 
\usepackage{multirow, array}
\usepackage[utf8]{inputenc} 
\usepackage[style=numeric,backend=bibtex,sorting=none,doi=true]{biblatex}
\addbibresource{biblio}
\usepackage{academicons}
\usepackage{svg}
\usepackage{orcidlink}
\usepackage{afterpage}
\usepackage{longtable}
\usepackage[table]{xcolor}
\usepackage{colortbl}
\usepackage{subcaption} % Package pour les sous-figures
\usepackage{hyperref}
\usepackage{cleveref}

\usepackage{appendixnumberbeamer}


% Définition des couleurs personnalisées pour les sections
\definecolor{mysectioncolor}{RGB}{255, 255, 255} % Couleur pour la section courante
\definecolor{myothersectionscolor}{RGB}{255, 255, 255} % Couleur pour les autres sections BLANC

\definecolor{blanc}{RGB}{255, 255, 255}
\definecolor{bleuNRJLAB}{RGB}{20,67,100}
\definecolor{bleuNRJLABfonce}{RGB}{20,67,100}
\definecolor{orangeNRJLAB}{RGB}{245,155,27}


\usecolortheme[named=bleuNRJLAB]{structure}
% Configuration du pied de page pour afficher le numéro de page à droite
%\setbeamertemplate{footline}[frame number]


% Configuration du pied de page
\setbeamertemplate{footline}{
	\begin{beamercolorbox}[wd=\paperwidth,ht=2.25ex,dp=1ex,leftskip=1em,rightskip=1em]{author in head/foot}
		\usebeamerfont{author in head/foot}\insertshortauthor\hfill\insertshortinstitute\hfill\insertframenumber{}/\inserttotalframenumber
	\end{beamercolorbox}
}

\newenvironment{blockorange}[1]{%
	\setbeamercolor{block title example}{bg=orangeNRJLAB,fg=white}%
	\setbeamercolor{block body example}{bg=orangeNRJLAB!10,fg=black}%
	\begin{exampleblock}{#1}}{\end{exampleblock}}

\newenvironment{blockbleu}[1]{%
	\setbeamercolor{block title example}{bg=bleuNRJLAB,fg=white}%
	\setbeamercolor{block body example}{bg=bleuNRJLAB!10,fg=black}%
	\begin{exampleblock}{#1}}{\end{exampleblock}}

\setbeamercolor{itemize item}{fg=black}


% Définir les marges de gauche et de droite
\setbeamersize{text margin left=0.5cm, text margin right=0.5cm}

% Modifier la taille de la police pour le headline
%\setbeamerfont{subsection in head/foot}{size=\small }

\usepackage[scaled]{helvet} % Charge la police Helvetica
\renewcommand{\familydefault}{\sfdefault} % Définit la police par défaut comme étant sans-serif (Helvetica)


% Définir une nouvelle commande pour les puces en noir et de taille spécifiée
\setbeamertemplate{itemize item}{\raisebox{-0.31em}{\textcolor{black}{\huge\textbullet}}}


% Informations sur le document
\title{\textbf{\textsf{Inter-comparaison entre les données d'observation du réseau IOS-net et les estimations satellitaires de SARAH-3}}}
\author{GRONDIN Erwan}
\institute{Université de La Réunion}
\titlegraphic{\includegraphics[width=0.4\textheight]{img/logo_UFRST}\quad\quad\quad \includegraphics[width=0.4\textheight]{img/logo_osu_r}\quad\quad\quad\includegraphics[width=0.4\textheight]{img/energy_labb}}
\date{\today}

\begin{document}
	

	
\begin{frame}
\textbf{STAGE M1}: 2 mois effectués / 2 mois restants\\
\textbf{Encadrants} : Dr.MOREL Béatrice et Dr.GRONDIN Dominique
\titlepage
\thispagestyle{empty} % Pour ne pas afficher l'en-tête sur cette page
\end{frame}

\begin{frame}
	\frametitle{Sommaire}
	\tableofcontents[]
	\thispagestyle{empty}
\end{frame}

\section{Introduction}
\begin{frame}
	\frametitle{Introduction}
	\begin{minipage}[t][0.4\textheight][t]{\textwidth}
		\begin{minipage}[t][\textheight][t]{0.48\textwidth}
			\begin{blockorange}{Contexte sur la Ressource Solaire (\textbf{SSR})}
				\begin{itemize}
					\setlength{\itemsep}{0.5pt} % Modification de l'espacement entre les items
					\item Ressource \textbf{abondante}
					\item Effet sur la variabilité \textbf{climatique}, \textbf{atmosphérique} et \textbf{hydrologique}.
					\item Permet de diversifier le mixe énergétique des \textbf{îles tropicales} et de diminuer les gaz à effet de serre (\textbf{GES}).
				\end{itemize}
			\end{blockorange}
		\end{minipage}
		\hfill
		\begin{minipage}[t][\textheight][t]{0.48\textwidth}
			\begin{blockorange}{\'Etat de l'art}
				\begin{itemize}
					\setlength{\itemsep}{0.5pt} % Modification de l'espacement entre les items
					\item \textbf{2023} : Christella IGIHOZO travaille sur l'inter-comparaison des données de SARAH-3 sur 14 stations du réseau IOS-net.
				\end{itemize}
			\end{blockorange}
		\end{minipage}
	\end{minipage}
	\vfill
	\begin{minipage}[t][0.58\textheight][t]{\textwidth}
		\begin{blockbleu}{Liens avec le Projet SoCooMa}
			\begin{itemize}
				\setlength{\itemsep}{0.5pt} % Modification de l'espacement entre les items
				\item \textbf{SoCooMa} : Concentration solaire pour cuisiner à Mafate (\textit{Solar Concentration for Cooking in Mafate})
				\item \textbf{Objectif} :  Estimation fine du rayonnement solaire en relief accidenté.
				\item \textbf{Tache} : inter-comparaison des estimations satellitaires et des mesures aux deux nouvelles stations de Mafate ainsi qu’aux stations PIMENT nouvellement intégrées dans le réseau IOS-net.
			\end{itemize}
		\end{blockbleu}
	\end{minipage}
\end{frame}

\section{Données}


\begin{frame}
	\frametitle{Jeu de données}
	\begin{columns}[T] % [T] ensures correct vertical alignment
		\begin{column}{0.48\linewidth} % Left column
			\centering
			\begin{blockbleu}{Caractéristiques des données IOS-net}
				\begin{itemize}
					\setlength{\itemsep}{0.5pt} % Modification de l'espacement entre les items
					\item \textbf{Stations} : 39 stations
					\item \textbf{Temporal grid} : 01/12/2008 - 01/04/2024
					\item \textbf{Product} : GHI, DHI, DNI, TA, PA, WD,WS
				\end{itemize}
			\end{blockbleu}
			\includegraphics[width=\linewidth]{img/SWIO_station}
		\end{column}
		\begin{column}{0.48\linewidth} % Right column
			\centering
			\begin{blockorange}{Caractéristiques des données SARAH-3}
				\begin{itemize}
					\setlength{\itemsep}{0.5pt} % Modification de l'espacement entre les items
					\item \textbf{Spatial grid } : ±65° de longitude / ±65° de latitude
					\item \textbf{Spatial precision } : 0,05° x 0,05°
					\item \textbf{Temporal grid} : 01/01/1983 - 01/04/2024
					\item \textbf{Product} : CAL, DAL, DNI, PAR , SDU, SID and SIS
				\end{itemize}
			\end{blockorange}
			\includegraphics[width=0.8\linewidth]{img/exemple_cm_saf_1}
		\end{column}
	\end{columns}
\end{frame}


\section{Méthodologie}
\subsection{IOS-net data}
\begin{frame}
	\frametitle{Importation des données}
	\begin{columns}[T] % [T] ensures correct vertical alignment
		\begin{column}{0.3\linewidth} % Left column
			\begin{blockorange}{Code python}
				\begin{itemize}
					\setlength{\itemsep}{0.5pt} % Modification de l'espacement entre les items
					\item Récupérer tous les liens \textbf{netCDF} avec requêtes \textit{HTTP} et \textit{BeautifulSoup}
					\item Sauvegarder les liens dans  une \textbf{liste} structurée
					\item Conversion de tous les fichiers netCDF en \textbf{Dataframe} avec \textit{xarray}.
					\item \textbf{Sélection} des données utiles (GHI,DHI,DNI et timestamp)
				\end{itemize}
			\end{blockorange}
		\end{column}
		\begin{column}{0.68\linewidth} % Center column
			\includegraphics[width=\linewidth]{img/ios_net_TDS}\\
			\includegraphics[width=\linewidth]{img/all_name_var}
		\end{column}
	\end{columns}
\end{frame}





\begin{frame}
	\frametitle{Estimation du DNI}
	\begin{columns}[T] % [T] ensures correct vertical alignment
		\begin{column}{0.47\linewidth} % Left column
			\begin{blockbleu}{\'Estimation du DNI ($W/m^2$)}
				\begin{equation}
					DNI = \frac{1}{\mu_0}(GHI - DHI)
				\end{equation}
				
				Avec $\mu_0$ le cosinus du l'angle zenithale
				
			\end{blockbleu}
			\begin{blockorange}{Considération de Pvlib}
				\begin{itemize}
					\setlength{\itemsep}{0.5pt} % Modification de l'espacement entre les items
					\item Latitude
					\item Longitude
					\item Altitude
					\item Time Zone
					\item Timestamp
				\end{itemize}
			\end{blockorange}
		\end{column}
		\begin{column}{0.47\linewidth} % Center column
			\includegraphics[width=\linewidth]{img/comparison_XY_of_CHP1_dni_ground_BSRN}
		\end{column}
	\end{columns}
\end{frame}


\begin{frame}
	\frametitle{Contrôle de qualité}
	\small % Reduce font size in this slide
	\vspace{0.1cm}
	\begin{center}
		\textbf{Limites physiques} avec $S_a$ la radiation solaire extraterrestre
	\end{center}
	
	\begin{columns}[T] % [T] ensures correct vertical alignment
		\begin{column}{0.3\linewidth} % Left column
			\includegraphics[width=\linewidth]{img/flow_data/ghi/amitie}\\[1 pt]
			\begin{center}
				\textbf{QC1- GHI-($W/m^2$)}
			\end{center}
			\begin{equation}
				S_a \times 1.5 \times \mu_0^{1.2} + 100
			\end{equation}
		\end{column}
		\begin{column}{0.3\linewidth} % Center column
			\includegraphics[width=\linewidth]{img/flow_data/dhi/amitie}\\[1pt]
			\begin{center}
				\textbf{QC2- DHI-($W/m^2$)}
			\end{center}
			\begin{equation}
				S_a \times 0.95 \times \mu_0^{1.2} + 50
			\end{equation}
		\end{column}
		\begin{column}{0.3\linewidth} % Right column
			\includegraphics[width=\linewidth]{img/flow_data/dni/amitie}\\[1pt]
			\begin{center}
				\textbf{QC3- DNI-($W/m^2$)}
			\end{center}
			\begin{equation}
				S_a
			\end{equation}
		\end{column}
	\end{columns}
\end{frame}

\begin{frame}
	\frametitle{Moyenne temporelle}
	
	\begin{minipage}[c][\textheight][t]{0.58\textwidth}
		\centering
		\includegraphics[height=0.85\textheight]{img/comparison_mean_method_at_BSRNof_CHP1_dni_ground_URBSRN}
	\end{minipage}
	\hfill
	\begin{minipage}[c][\textheight][t]{0.37\textwidth}
		\centering
		\includegraphics[height=0.85\textheight]{img/mean_fonction}
	\end{minipage}
\end{frame}

\subsection{SARAH-3 data}
\begin{frame}
	\frametitle{Importation des données SARAH-3}
	
	\begin{minipage}[c][\textheight][t]{0.48\textwidth}
		\vspace{1cm}
		\begin{minipage}[c][0.3\textheight][t]{\textwidth}
			
			\begin{table}
				\centering
				\renewcommand{\arraystretch}{1} % Ajustement de la hauteur des lignes
				\footnotesize
				\small
				\rowcolors{1}{gray!25}{white}
				\begin{tabular}{>{\fontsize{5}{6}\selectfont\bfseries}l >{\fontsize{5}{6}\selectfont}c >{\fontsize{5}{6}\selectfont}c}
					\toprule
					\textbf{ID}  &\textbf{GHI} &\textbf{DNI}\\
					\midrule
					Product group & Climate Data Records & Climate Data Records \\
					Product family & SARAH ed. 3.0 & SARAH ed. 3.0 \\
					Product name & SIS & DNI \\
					Temporal resolution & Instantaneous & Instantaneous \\
					
					\bottomrule
				\end{tabular}
				\label{param_upload}
			\end{table}
		\end{minipage}
		\vfill
		\begin{minipage}[c][0.65\textheight][t]{\textwidth}
			\begin{figure}[t]
				\centering
				\footnotesize
				\includegraphics[width=\textwidth]{img/diagrame_code}
				\caption{Diagram of the interface of gnu-MAGIC / SPECMAGIC to the atmospheric input and the satellite observations. (Source : \href{https://www.cmsaf.eu/SharedDocs/Literatur/document/2023/saf_cm_dwd_atbd_sarah_3_5_pdf.pdf?__blob=publicationFile}{EUMETSAT})}
				\label{fig_diagramecode}
			\end{figure}
		\end{minipage}
	\end{minipage}
	\hfill
	\begin{minipage}[c][\textheight][t]{0.48\textwidth}
		\centering
		\includegraphics[width=0.8\textwidth]{img/sarah_show/SISdm20240101}\\
		\textbf{CM-SAF SARAH-3 representation of GHI at SOOI area }\\
		\includegraphics[width=0.8\textwidth]{img/sarah_show/DNIdm20240101}\\
		\textbf{CM-SAF SARAH-3 representation of DNI at SOOI area }
	\end{minipage}
\end{frame}



\subsection{Comparaison des données}

\begin{frame}
	\frametitle{\'Etapes pour la comparaison entre les données IOS-net et SARAH-3}
	\begin{columns}[b] % [T] ensures correct vertical alignment
		\begin{column}{0.33\linewidth} % Left column
			\centering
			\includegraphics[width=\textwidth]{img/adapted_station_REUNION}\\
			\textbf{Adaptation} des données géographiques
		\end{column}
		\begin{column}{0.33\linewidth} % Center column
			\centering
			\includegraphics[width=\linewidth]{img/effect_polyorder.png}\\
			\includegraphics[width=\linewidth]{img/effect_window_length.png}\\
			Application du \textbf{filtre Savitzky-Golay}
		\end{column}
		\begin{column}{0.33\linewidth} % Right column
			\centering
			\begin{table}[!h]
				\centering
				\renewcommand{\arraystretch}{1.2} % Ajustement de la hauteur des lignes
				\footnotesize
				\small
				\rowcolors{1}{gray!25}{white}
				\begin{tabular}{>{\footnotesize\bfseries}l | >{\footnotesize}c >{\footnotesize}c}
					\toprule
					\textbf{Daily type}  &\textbf{4 classes} &\textbf{5 classes}\\
					\midrule
					
					Clear sky & \tikz[baseline=-0.5ex]\node[draw,minimum width=1em,minimum height=1em] (box){$\checkmark$}; & \tikz[baseline=-0.5ex]\node[draw,minimum width=1em,minimum height=1em] (box){$\checkmark$}; \\
					Overcast & \tikz[baseline=-0.5ex]\node[draw,minimum width=1em,minimum height=1em] (box){$\checkmark$}; & \tikz[baseline=-0.5ex]\node[draw,minimum width=1em,minimum height=1em] (box){$\checkmark$}; \\
					AM clear & \tikz[baseline=-0.5ex]\node[draw,minimum width=1em,minimum height=1em] (box){$\checkmark$}; & \tikz[baseline=-0.5ex]\node[draw,minimum width=1em,minimum height=1em] (box){$\checkmark$}; \\
					PM clear & \tikz[baseline=-0.5ex]\node[draw,minimum width=1em,minimum height=1em] (box){$\checkmark$}; & \tikz[baseline=-0.5ex]\node[draw,minimum width=1em,minimum height=1em] (box){$\checkmark$}; \\
					Random & \tikz[baseline=-0.5ex]\node[draw,minimum width=1em,minimum height=1em] (box){}; & \tikz[baseline=-0.5ex]\node[draw,minimum width=1em,minimum height=1em] (box){$\checkmark$}; \\
					
					\bottomrule
				\end{tabular}
			\end{table}
			\textbf{Classification} des cycles journaliers
			
		\end{column}
	\end{columns}
\end{frame}




\section{Résultats}
\subsection{Comparaison DNI}
\begin{frame}
	\frametitle{Résultats du site du BSRN}
	\begin{columns}[T] % [T] ensures correct vertical alignment
		\begin{column}{0.49\linewidth} % Left column
			\centering
			\includegraphics[width=\linewidth]{img/comparison_curve_of_CHP1_dni_ground_URBSRN.png}
		\end{column}
		\begin{column}{0.49\linewidth} % Center column
			\centering
			\includegraphics[width=\linewidth]{img/comparison_XY_of_CHP1_dni_ground_URBSRN.png}
		\end{column}
	\end{columns}
\end{frame}


\subsection{Comparaison GHI}
\begin{frame}
	\frametitle{Résultats du site du BSRN}
	\begin{columns}[T] % [T] ensures correct vertical alignment
		\begin{column}{0.49\linewidth} % Left column
			\centering
			\includegraphics[width=\linewidth]{img/comparison_curve_of_CMP22_ghi_URBSRN.png}
		\end{column}
		\begin{column}{0.49\linewidth} % Center column
			\centering
			\includegraphics[width=\linewidth]{img/comparison_XY_of_CMP22_ghi_URBSRN.png}
		\end{column}
	\end{columns}
\end{frame}

\subsection{Indicateur de précision}
\begin{frame}
	\frametitle{Résultats pour toutes les stations (daily mean)}
	\begin{columns}[T] % [T] ensures correct vertical alignment
		\begin{column}{0.49\linewidth} % Left column
			\centering
			\includegraphics[width=\textwidth]{img/stat_indicator_of_DNI}
		\end{column}
		\begin{column}{0.49\linewidth} % Center column
			\centering
			\includegraphics[width=\textwidth]{img/stat_indicator_of_GHI}
		\end{column}
	\end{columns}
\end{frame}

\section{Conclusion et Discussion}
\begin{frame}
	\frametitle{Conclusion et Discussion}
	\begin{columns}[T] % [T] ensures correct vertical alignment
		\begin{column}{0.32\linewidth} % Left column
			\centering
			\begin{blockorange}{Discussions}
				\textbf{Erreurs :}
				\begin{itemize}
					\item Non prise en compte de \textbf{données moyennées par CM-SAF}
					\item Résolution spatiale de \textbf{SARAH-3}(25 km²)
					\item Estimation du \textbf{DNI} 
					\item Estimation du \textbf{SZA} 
				\end{itemize} 
			\end{blockorange}
		\end{column}
		\begin{column}{0.32\linewidth} % Center column
			\centering
			\begin{blockbleu}{Conclusion}				
				\begin{itemize}
					\item Le \textbf{GHI} est mieux estimé que le DNI.
					\item \textbf{92\%} des données de DNI sont sous-estimées.
					\item \textbf{40\%} des données de GHI sont surestimées .
					\item Il en ressort que \textbf{SARAH-3} est un outil intéressant pour l'estimation de la ressource solaire dans la zone SOOI. 
				\end{itemize}
			\end{blockbleu}
		\end{column}
		\begin{column}{0.32\linewidth} % Center column
			\centering
			\begin{blockorange}{Perspectives}
				\begin{itemize}
					\setlength{\itemsep}{1pt} % Modification de l'espacement entre les items
					\item Appliquer les filtres et les classifications pour les 38 autres stations afin d'\textbf{améliorer les estimations} de SARAH-3.
					\item Réussir une \textbf{descente d'échelle} pour la résolution du satellite SARAH-3
					\item Considération du relief (Site Adaptation)
				\end{itemize}
			\end{blockorange}
		\end{column}
	\end{columns}
	\begin{center}
		\includegraphics[width=0.4\textheight]{img/logo_UFRST}
		\quad\quad\quad
		\includegraphics[width=0.4\textheight]{img/logo_osu_r}
		\quad\quad\quad
		\includegraphics[width=0.4\textheight]{img/energy_labb}
	\end{center}
\end{frame}


% Commencez les annexes
\appendix
\section{Annexes}
\subsection{Monthly}
\begin{frame}
	\frametitle{Résultats pour toutes les stations}
	\begin{columns}[T] % [T] ensures correct vertical alignment
		\begin{column}{0.49\linewidth} % Left column
			\centering
			\includegraphics[width=\textwidth]{img/stat_indicator_of_DNI_monthly}
		\end{column}
		\begin{column}{0.49\linewidth} % Center column
			\centering
			\includegraphics[width=\textwidth]{img/stat_indicator_of_GHI_monthly}
		\end{column}
	\end{columns}
\end{frame}

\subsection{Yearly}
\begin{frame}
	\frametitle{Résultats pour toutes les stations}
	\begin{columns}[T] % [T] ensures correct vertical alignment
		\begin{column}{0.49\linewidth} % Left column
			\centering
			\includegraphics[width=\textwidth]{img/stat_indicator_of_DNI_yearly}
		\end{column}
		\begin{column}{0.49\linewidth} % Center column
			\centering
			\includegraphics[width=\textwidth]{img/stat_indicator_of_GHI_yearly}
		\end{column}
	\end{columns}
\end{frame}

\subsection{Information sur les stations}
\begin{frame}
	\frametitle{Les mesures in situ IOS-net}
	% Première moitié de la page en hauteur, et toute la largeur de la frame
	\begin{minipage}[t][0.2\textheight][t]{\textwidth}
		\begin{blockorange}{Appareils de mesures}
			\small
			\begin{itemize}
				\setlength{\itemsep}{0.5pt} % Modification de l'espacement entre les items
				\item \textbf{SPN1 / CMP22} : Pour les données de GHI et DHI
				\item \textbf{CHP1} : Pour les données de DNI (BSRN)
			\end{itemize}
		\end{blockorange}
	\end{minipage}
	
	\vfill
	
	% Deuxième moitié de la page, divisée en deux colonnes
	\begin{minipage}[c][0.8\textheight][c]{0.45\textwidth}
		\begin{blockbleu}{\'Equipement des stations}
			\small
			\begin{itemize}
				\setlength{\itemsep}{0.5pt} % Modification de l'espacement entre les items
				\item Pyranomètre SPN1
				\item Transmetteur météorologique WXT530
				\item Radiomètre UV
				\item Centrale d’acquisition
				\item \'Eolienne
				\item Panneau photovoltaïque
				\item Régulateur de charge
				\item Batterie de plomb 20 AH
			\end{itemize}
		\end{blockbleu}
	\end{minipage}
	\hfill
	\begin{minipage}[c][0.8\textheight][c]{0.45\textwidth}
		\includegraphics[trim=3cm 0cm 3cm 1cm, clip, width=1\linewidth]{img/SWIO_station}
	\end{minipage}
	
\end{frame}

\subsection{Données de l'étude}
\begin{frame}
	\frametitle{Données utiles}
	\begin{columns}[T] % [T] ensures correct vertical alignment
		\begin{column}{0.48\linewidth} % Left column
			\begin{itemize}
				\item Irradiance Horizontale Globale (GHI) en $W/m^2$.
				\item Irradiance Normale Directe (DNI) en $W/m^2$ .
			\end{itemize}
			\begin{blockorange}{Considération de l'étude}
				\textbf{Données}: 
				\begin{itemize}
					\item 39 stations \textbf{IOS-net}
					\item \textbf{SARAH-3} dans la zone SOOI
				\end{itemize}
				\textbf{Plage temporelle} : 01/12/2008-01/04/2024
				\begin{itemize}
					\item  01/12/2008-01/04/2024
					\item 02H-14H (données du jour)
				\end{itemize}
			\end{blockorange}
		\end{column}
		\begin{column}{0.48\linewidth} % Right column
			\centering
			\includegraphics[width=0.7\linewidth]{img/panneau_solaire}\\
			\textbf{Panneau solaire (GHI)}\\
			\includegraphics[width=0.7\linewidth]{img/four_solaire}\\
			\textbf{Système à concentration solaire (DNI)}
		\end{column}
	\end{columns}
\end{frame}

\begin{frame}
	\frametitle{Comparaison des méthodes de moyennes}
	\centering
	\includegraphics[width=\linewidth]{img/comparison_mean_method}
	\begin{minipage}[c][0.55\textheight][t]{\textwidth}
		\begin{blockbleu}{Mean comparison}
			\begin{itemize}
				\setlength{\itemsep}{0.5pt} % Modification de l'espacement entre les items
				\item pandas mean : \texttt{resample('D').mean()}
				\item CM-SAF mean :
				\begin{equation}
					SIS_{DA} = SIS_{CLSDA} \frac{\sum\limits_{i=1}^{n}SIS_i}{\sum\limits_{i=1}^{n}SIS_{CLS_i}}
					\label{SIS_average}
				\end{equation}
				Avec $SIS_{DA}$ la moyenne journalière du SIS,  $SIS_{CLSDA}$ la moyenne journalière du SIS en ciel clair, $SIS_i$ valeur instantanée du SIS par les images satellites et $SIS_{CLS_i}$ est le SIS calculé correspondant au ciel clair.
			\end{itemize}
		\end{blockbleu}
	\end{minipage}
\end{frame}
	
\end{document}
